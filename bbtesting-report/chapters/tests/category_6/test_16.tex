\subsection{Testing for HTTP Splitting/Smuggling - OTG-INPVAL-016}
\subsubsection{BANK-APP}
\begin{longtable}[l]{ p{2.3cm} | p{.79\linewidth} }\hline
    & \textbf{BANK-APP} \\ \hline
    \textbf{Observation} & The application does not use the \code{Location} header with GET parameters. \\
    \textbf{Discovery} & With ZAP all HTTP headers were examined. The results showed that no \code{Location} header was used in conjunction with GET parameters. \\
    \textbf{Likelihood} & N/A \\
    \textbf{Impact} & N/A \\
    \textbf{Recommen\-dations} & N/A \\ \hline
    \textbf{CVSS} & N/A \\ \hline
\end{longtable}

\subsubsection{SecureBank}
\begin{longtable}[l]{ p{2.3cm} | p{.79\linewidth} }\hline
    & \textbf{SecureBank} \\ \hline
    \textbf{Observation} & The application does not use the \code{Location} header with GET parameters. \\
    \textbf{Discovery} & With ZAP all HTTP headers were examined. The results showed that no \code{Location} header was used in conjunction with GET parameters. \\
    \textbf{Likelihood} & N/A \\
    \textbf{Impact} & N/A \\
    \textbf{Recommen\-dations} & N/A \\ \hline
    \textbf{CVSS} & N/A \\ \hline
\end{longtable}

\subsubsection{Comparison}
Both applications do not use 302 requests with the \code{Location} header in conjunction with GET parameters. Therefore HTTP splitting/smuggling is not possible.
\clearpage