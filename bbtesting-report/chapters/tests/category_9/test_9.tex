\subsection{Test Upload of Malicious Files - OTG-BUSLOGIC-009}
\subsubsection{BANK-APP}
\begin{longtable}[l]{ p{2.3cm} | p{.79\linewidth} }\hline
    & \textbf{BANK-APP}
    \hfill CVSS Score: 10 \progressbar[filledcolor=red]{1}
    \\ \hline
    \textbf{Observation} & There is no restriction on the type of files to be uploaded in the Transaction page. Hence this can be used to upload malicious files. \\
    \textbf{Discovery} &
         No tools were used to discover this vulnerability and manual testing was performed. Steps followed are as given below.
         \begin{itemize}
             \item Login as a customer and go to the "New Transaction" interface.
             \item Upload a file with name containing parenthesis like \enquote{xxx(1).pdf}, then the application returns a success message, though the transaction is not reflected under \enquote{View Transactions}. This was tried with other file extensions and the output was the same. Though it cannot be guaranteed, it is very likely that the file was uploaded successfully. Similarly, malicious files of type .exe, .bat, .py etc. could be uploaded to trigger attacks.
             Since the file is not visible in the \code{http://<IP-address>/secure-coding/app/} path, it indicates that the uploaded file gets deleted after parsing is complete. Hence we were not able to simulate a successful attack. However, it is possible for the attacker to block the server and perform a malicious action through the file before it is deleted.
         \end{itemize}
    \\
    \textbf{Likelihood} & Technical knowledge may be required to execute the attack as it is required to perform malicious actions before the file is deleted and requires appropriate timing. \\
    \textbf{Impact} & With this attack, any file may be uploaded and complete control over the system could be gained. \\
    \textbf{Recommen\-dations} & It is strictly recommended to restrict all irrelevant file-types.\\ \hline
    \textbf{CVSS} &
        \begin{tabular}[t]{@{}l | l}
            Attack Vector           & \textcolor{red}{Network} \\
            Attack Complexity       & \textcolor{red}{Low} \\
            Privileges Required     & \textcolor{red}{None} \\
            User Interaction        & \textcolor{red}{None} \\
            Scope                   & \textcolor{red}{Changed} \\
            Confidentiality Impact  & \textcolor{red}{High} \\
            Integrity Impact        & \textcolor{red}{High} \\
            Availability Impact     & \textcolor{red}{High}
        \end{tabular}
    \\ \hline
\end{longtable}

\subsubsection{SecureBank}
\begin{longtable}[l]{ p{2.3cm} | p{.79\linewidth} }\hline
    & \textbf{SecureBank}
    \\ \hline
    \textbf{Observation} & In the application we observed that upload of malicious files is not possible since upload is restricted to files of type plain text only. \\
    \textbf{Discovery} & N/A \\
    \textbf{Likelihood} & N/A \\
    \textbf{Impact} & N/A \\
    \textbf{Recommen\-dations} & N/A \\ \hline
    \textbf{CVSS} & N/A
    \\ \hline
\end{longtable}

\subsubsection{Comparison}
SecureBank is better than BANK-APP as it does not allow the upload of any files other than plain text. Hence the possibility of uploading malicious files is ruled out.
\clearpage