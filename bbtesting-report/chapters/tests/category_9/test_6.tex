\subsection{Testing for the Circumvention of Work Flows - OTG-BUSLOGIC-006}
\subsubsection{BANK-APP}
\begin{longtable}[l]{ p{2.3cm} | p{.79\linewidth} }\hline
    & \textbf{BANK-APP}
    \hfill CVSS Score: 8.3 \progressbar[filledcolor=red]{0.83}
    \\ \hline
    \textbf{Observation} & It is possible to perform an action that is not acceptable per the business logic work-flow and this vulnerability has been observed in the New Transaction page. A customer can steal money from other accounts and thus, effectively increase balance in his/her own account. \\
    \textbf{Discovery} &
        No specific tool was required to discover this vulnerability, it was encountered by manual testing. Steps are as follows.
           \begin{itemize}
     	      \item Login as a Customer and click on "New Transaction.

     	      \item Enter the valid details in the Recipient Account \& TAN fields, but provide a negative value in the Amount field. Note that the transaction is auto-approved and the account is credited with the entered amount. The Recipient Account is debited with the same amount.
           \end{itemize}
    \\
    \textbf{Likelihood} & This vulnerability does not require any technical skills. Any customer who is logged in to the bank can perform this action. It is exploitable remotely via the web interface and via the batch file functionality. Likelihood is high. Also, guessing the Account number to enter in the Recipient ID field is not difficult either, as they are sequential and a Brute-force method is quite easy. \\
    \textbf{Impact} & The user can transfer infinite amounts of money from other accounts into his own, by entering negative values in the Amount field. In other words, an attacker can gain complete control over other accounts and steal the entire money. It can even result in a Denial of Service(DOS) for the victim as it will not be possible for him/her to perform any transactions due to insufficient funds. \\
    \textbf{Recommen\-dations} & User inputs should be validated against improper values in order to avoid such scenarios. \\ \hline
    \textbf{CVSS} &
        \begin{tabular}[t]{@{}l | l}
            Attack Vector           & \textcolor{red}{Network} \\
            Attack Complexity       & \textcolor{red}{Low} \\
            Privileges Required     & \textcolor{BurntOrange}{Low} \\
            User Interaction        & \textcolor{red}{None} \\
            Scope                   & \textcolor{Green}{Unchanged} \\
            Confidentiality Impact  & \textcolor{BurntOrange}{Low} \\
            Integrity Impact        & \textcolor{red}{High} \\
            Availability Impact     & \textcolor{red}{High}
        \end{tabular}
    \\ \hline
\end{longtable}

\subsubsection{SecureBank}
\begin{longtable}[l]{ p{2.3cm} | p{.79\linewidth} }\hline
    & \textbf{SecureBank}
    \\ \hline
    \textbf{Observation} & In the application, there is a restriction on transfer of negative funds. Hence there is no possibility of transferring money from others account into one's own.\\
    \textbf{Discovery} & On performing the same steps as described above, an error message was displayed to enter a right value for the Amount field. \\
    \textbf{Likelihood} & N/A \\
    \textbf{Impact} & N/A \\
    \textbf{Recommen\-dations} & N/A \\ \hline
    \textbf{CVSS} & N/A
    \\ \hline
\end{longtable}

\subsubsection{Comparison}
SecureBank restricts entry of negative values in the Amount field, thus making the application more secure than BANK-APP, in this aspect.
\clearpage