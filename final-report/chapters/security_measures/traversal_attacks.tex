\section{Directory Traversal/File Include Attacks}

\subsection{Attack Description}
Using and managing files is common to web applications. Traditionally, web servers and web applications implement authentication mechanisms to control access to files and resources. Web servers try to confine files related to users inside a \enquote{root directory} or \enquote{web document root}, which represents a physical directory on the file system. Users have to consider this directory as the base directory into the hierarchical structure of the web application.

The purpose of defining privileges is to identify which users or groups are supposed to be able to access, modify, or execute a specific file on the server. These mechanisms are designed to prevent malicious users from accessing sensitive files or to avoid the execution of system commands.

By exploiting this kind of vulnerability, an attacker is able to read directories or files which are normally inaccessible, access data outside the web document root, execute arbitrary code or system commands or include scripts and other kinds of files from external websites.

\subsection{Countermeasure}
In order to safeguard the application from this vulnerability, following measures are undertaken in the \textbf{SecureBank} application.
\begin{itemize}
\item \textbf{Disabling Directory Listing} - This has been done by setting \textit{Options -Indexes} in the \textit{src/.htaccess} file. Hence it is not possible to access any source code or download files by accessing files/folders. The detailed configuration options can also be found in the configuration file for the application - \textit{secure-bank.conf}.
\end{itemize}

\clearpage