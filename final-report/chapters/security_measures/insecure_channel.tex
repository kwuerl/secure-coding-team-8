\section{Sensitive data sent via Unencrypted Channel}

\subsection{Attack Description}
Sensitive data must be protected when it is transmitted through the network. Some examples for sensitive data are Credentials, PINs, Session identifiers, Tokens, Cookies etc.
If the application transmits sensitive information via unencrypted channels like \textbf{HTTP}, it is considered a security risk.

\subsection{Countermeasure}
In order to safeguard the application from this vulnerability, following measures are undertaken in the \textbf{SecureBank} application.
\begin{itemize}
\item \textbf{Usage of HTTPS} - It has been found that all accesses to the application are via
\textit{HTTPS} i.e., secure \textbf{HTTP} connections. Unencrypted traffic over \textit{HTTP} is
also redirected to the encrypted version of the webpage.
\item \textbf{Usage of HSTS} - The HSTS (HTTP Strict Transport Security) header for \\ \code{Strict-Transport-Security} is set to \code{max-age= 60000; includeSubDomains}. The  configuration details for SSL and HSTS are found in the file \textit{secure-bank.conf}.
\end{itemize}

\clearpage