\section{Cross-Site scripting}

\subsection{Attack Description}
Cross-Site Scripting (XSS) attacks are a type of injection, in which malicious scripts are injected into otherwise benign and trusted web sites. XSS attacks occur when an attacker uses a web application to send malicious code, generally in the form of a browser side script, to a different unsuspecting end user. Flaws that allow these attacks to succeed are quite widespread and occur anywhere a web application uses input from a user within the output it generates without validating or encoding it. \\

The end user’s browser has no way to know that the script should not be trusted, and will execute the script. Because it thinks the script came from a trusted source, the malicious script can access any cookies, session tokens, or other sensitive information retained by the browser and used with that site. These scripts can even rewrite the content of the HTML page.

\subsection{Countermeasure}
In order to safeguard the application from this vulnerability, following measures are undertaken in the \textbf{SecureBank} application.
\begin{itemize}
\item \textbf{Sanitization of user inputs before further processing} -  \textit{Helper/ValidationHelper.php} and \textit{Helper/SanitizationHelper.php} implement functions for validation and sanitization of user input. PHP functions such as \textit{htmlspecialchars}, \textit{filter\_var} and \textit{preg\_match} are employed for input sanitiza-
tion; leading to HTML tags being rendered as plain text. These \textit{Helper} classes are used whenever forms are rendered in the web interface.
\end{itemize}

\clearpage