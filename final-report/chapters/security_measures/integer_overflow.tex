\section{Integer Overflow}

\subsection{Attack Description}
An integer overflow condition exists when an integer, which has not been properly sanity checked, is used in the determination of an offset or size for memory allocation, copying, concatenation, or similarly. If the integer in question is incremented past the maximum possible value, it may wrap to become a very small, or negative number, therefore providing a very incorrect value.

\subsection{Countermeasure}
In order to safeguard the application from Integer overflows, following measures are undertaken in the \textbf{SecureBank} application.
\begin{itemize}
\item \textbf{Sanitization of user inputs before further processing} -  \textit{Helper/ValidationHelper.php} and \textit{Helper/SanitizationHelper.php} implement functions for validation and sanitization of user input. These functions are used whenever forms are rendered in the web interface.
\item \textbf{Data validation} - This is done by enforcing appropriate checks to validate data during specific operations such as Account Balance Initialization by an employee, Single Transaction and also Batch transfers. This has been done both in PHP and C code.
\end{itemize}

\clearpage