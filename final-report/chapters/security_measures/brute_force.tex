\section{Brute Force Attacks}

\subsection{Attack Description}
A brute-force attack is an attempt to discover a password by systematically trying every possible combination of letters, numbers, and symbols until the one correct combination that works is discovered.
Though it is always possible for an attacker to discover a password through a brute-force attack, the downside is that it could take years to find it. The possible combinations of passwords could increase enormously depending on the length and complexity. 

\subsection{Countermeasure}
In order to safeguard the application from Brute force attacks during login, following measures are undertaken in the \textbf{SecureBank} application.

\begin{itemize}
\item \textbf{Enforcement of Strong Password} - There is a restriction on the choice of passwords during registration. User needs to enter minimum 6 characters
which should contain atleast one uppercase character, one lowercase and one number. This reveals that passwords of users cannot be cracked easily.

\item \textbf{Password Encryption} - Before saving the password in the database, it has been encrypted using the \textit{crypt} function based on a random salt. So
even for users having the same password, the database entries will be different.

\item \textbf{Lockout Mechanism} - Logging in with a false password can be repeated 5 times before the
account is locked. In the file \textit{Auth/DBAuthProvider.php}, the account is checked for the number of login attempts. If it is greater than or equal to 4, it is set to 5 and the account is locked for 60 minutes.
\end{itemize}

\clearpage