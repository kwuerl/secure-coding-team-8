\section{Unauthorized Access to details of other customers - OTG-AUTHZ-004}

\subsection{Observation}
A test for this vulnerability was performed in the User page and no vulnerability has been found.

\subsection{Discovery}
No specific tool was used to test this vulnerability and manual testing was carried out.\\
Steps: \\
1. Login as a Customer. \\
2. Click on the Customer name next to the Logout button. \\
3. The profile and account details of the logged in customer are shown. \\
4. The URL in the address bar is of the form \\ \textit{"<IP-address>/secure-coding/public/view\_user.php?id=7"}. \\
5. Edit the id at the end to any other number, expecting to view details of a customer with that id. \\

However, the page always displays the details of the logged-in user. 
Even if the id parameter is removed from the URL, the details of the logged-in user are still visible.

\subsection{Likelihood}
Likelihood is high.
The attacker need not have any specialized skills to exploit this vulnerability. Any customer who is logged in to the bank can perform this action. 

\subsection{Impact}
If this vulnerabilty exists, then a customer can get hold of details pertaining to other customers such as Account number, current balance etc. Using the Account numbers, he/she can make infinite transactions with negative amounts, thus transferring money from the victim's account to his/her own. Knowing the current balance might further aid in performing transactions of calculated amounts, without acquiring the attention of the victim. 

\subsection{CVSS}
\begin{tabular}{l | l}
Attack Vector		& Network \\
Attack Complexity	& Low \\
Privileges Required & Low \\
User Interaction	& None \\
Scope				& Unchanged \\
Confidentiality		& High \\
Integrity			& None \\
Availability		& None
\end{tabular}

\subsection{Comparison with our application}
In our application, there is no URL to view a specific user. The user details are visible in Profile and Overview under the static URLs \\ 
\textit{"<IP>/profile"} and \textit{"<IP>/overview"} that do not contain a parameter. Hence there is no possibility of modifying the URL, thus making our application more secure, in this aspect.
\clearpage