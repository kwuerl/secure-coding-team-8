\subsection{Test Business Logic Data Validation - OTG-BUSLOGIC-001}

\subsubsection{Observation 1}
It has been found that it is possible to enter valid data and cause the application to behave differently due to a deviation in the business logic. One such vulnerabilities has been found in the Transaction page where it is possible to perform a transfer with amount 0.00.

\subsubsection{Discovery}
This vulnerability has been exposed through manual testing.\\
Steps: \\
1. Login as a Customer. \\
2. Click on the New Transaction button at the top. \\
3. In the form, enter the following values: \\
\begin{tabular}{l | l}
Recipient Account 	& Valid Account Number\\
TAN			& Valid TAN \\
Amount		& 0.00
\end{tabular} \\
4. Click on the Submit button. \\
5. The transaction is successful, as indicated by a message. \\
6. Click on the Transaction button on top to view the list of all transactions. \\
7. The last transaction of amount 0.00 is shown.

\subsubsection{Likelihood}
Likelihood is low.
The attacker need not have any technical knowledge to perform this action.

\subsubsection{Impact}
The recipient account shows a transaction of 0.00. This could lead him/her to think that it was a fake transaction. Additionally, it is possible to enter -0.00. This would lead the recipient to believe that his/her account has been hacked.

\subsubsection{CVSS}
\begin{tabular}{l | l}
Attack Vector		& Network \\
Attack Complexity	& Low \\
Privileges Required & Low \\
User Interaction	& None \\
Scope				& Unchanged \\
Confidentiality Impact		& Low \\
Integrity Impact			& Low \\
Availability Impact		& None
\end{tabular}

\subsubsection{Comparison with our application}
In our application, it is possible to enter 0.00 in the amount and perform a successful transfer. However, entering negative values is not allowed.

\subsubsection{Recommendation}
All invalid values such as negative and 0 amounts need to be restricted both on the client and server side of the application.

\subsubsection{Observation 2}
A similar vulnerability has been found in the Registration page where it is possible to register with any Email address. It is possible to successfully register and become a user without a valid email address, with the only exception of not being able to receive the TAN numbers.

\subsubsection{Discovery}
This vulnerability has been exposed through manual testing.\\
Steps: \\
1. Open the Registration page. \\
2. In the form, fill all the details and provide a non-existent email address, but adhering to the pattern of a valid email address. \\
3. Click on the Submit button. \\
4. Registration is successful, as indicated by a message. \\
Once the registration is approved by another user, it is possible to login to the account.

\subsubsection{Likelihood}
Likelihood is high.
The attacker need not have any technical knowledge to perform this action and can register with any random email address.

\subsubsection{Impact}
It is possible to gain access to the system with no valid email address. Once logged in, the user can take advantage to exploit other vulnerabilities with a few of them described in sections \ref{OTG-AUTHZ-002} and \ref{OTG-IDENT-003}.

\subsubsection{CVSS}
\begin{tabular}{l | l}
Attack Vector		& Network \\
Attack Complexity	& Low \\
Privileges Required & Low \\
User Interaction	& None \\
Scope				& Unchanged \\
Confidentiality Impact		& Low \\
Integrity Impact			& Low \\
Availability Impact		& None
\end{tabular}

\subsubsection{Comparison with our application}
The same behavior is reflected in our application too, where there is no check on the authenticity of the email address entered.

\subsubsection{Recommendation}
It would be better to have an activation link sent to the email address and only upon clicking of the link, registration could be considered as successful. Such a mechanism should be enforced to tackle the above vulnerability.

\clearpage