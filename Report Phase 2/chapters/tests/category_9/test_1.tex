\subsection{Test Business Logic Data Validation - OTG-BUSLOGIC-001}
\subsubsection{BANK-APP}
\begin{longtable}[l]{ p{2.3cm} | p{.79\linewidth} }\hline
    & \textbf{BANK-APP}
    \\ \hline
    \textbf{Observation} &
        It has been found that it is possible to enter valid data and cause the application to behave differently due to a deviation in the business logic. Two such vulnerabilities have been found and they are as follows.
        \begin{itemize}
            \item In the Transaction page, it is possible to perform a transfer with amount 0.00.
            \item In the Registration page, it is possible to sucessfully register with any Email address and become a user without a valid email address. The only exception is not being able to receive the TAN numbers.
        \end{itemize}
    \\
    \textbf{Discovery} &
        This vulnerability has been exposed through manual testing using the steps described below.
        \begin{itemize}
             \item Login as a Customer and click on the New Transaction button at the top.
             \item In the form, enter valid values for Recipient Account number \& TAN, but provide \enquote{0.00} in the Amount field. Click on Submit.
             \item The transaction is successful, as indicated by a message. Click on the Transaction button on top to view the list of all transactions. Notice that the last transaction of amount 0.00 is shown.
        \end{itemize}
    \\
    \textbf{Likelihood} & Likelihood is low. The attacker need not have any technical knowledge to perform this action. \\
    \textbf{Impact} &
        \begin{itemize}
            \item The recipient account shows a transaction of 0.00. This could lead him/her to think that it was a fake transaction. Additionally, it is possible to enter -0.00. This would lead the recipient to believe that his/her account has been hacked.
            \item It is possible to gain access to the system with no valid email address. Once logged in, the user can take advantage to exploit other vulnerabilities with a few of them described in sections \ref{OTG-AUTHZ-002} and \ref{OTG-IDENT-003}.
        \end{itemize}
    \\
    \textbf{Recommen\-dations} &
        \begin{itemize}
        \item All invalid values such as negative and 0 amounts need to be restricted both on the client and server side of the application.
        \item It would be better to have an activation link sent to the email address and only upon clicking of the link, registration could be considered as successful. Such a mechanism should be enforced to tackle the above vulnerability.
        \end{itemize}
    \\ \hline
    \textbf{CVSS} &
        \begin{tabular}[t]{@{}l | l}
            Attack Vector           & \textcolor{red}{Network} \\
            Attack Complexity       & \textcolor{red}{Low} \\
            Privileges Required     & \textcolor{BurntOrange}{Low} \\
            User Interaction        & \textcolor{red}{None} \\
            Scope                   & \textcolor{Green}{Unchanged} \\
            Confidentiality Impact  & \textcolor{BurntOrange}{Low} \\
            Integrity Impact        & \textcolor{BurntOrange}{Low} \\
            Availability Impact     & \textcolor{Green}{None}
        \end{tabular}
    \\ \hline
\end{longtable}
\clearpage
\subsubsection{SecureBank}
\begin{longtable}[l]{ p{2.3cm} | p{.79\linewidth} }\hline
    & \textbf{SecureBank}
    \\ \hline
    \textbf{Observation} &
         It has been found that it is possible to enter valid data and cause the application to behave differently due to a deviation in the business logic. Two such vulnerabilities have been found and they are as follows.
         \begin{itemize}
             \item In the Transaction page, it is possible to perform a transfer with amount 0.00.
             \item In the Registration page, it is possible to sucessfully register with any Email address and become a user without a valid email address. The only exception is not being able to receive the TAN numbers.
         \end{itemize}
    \\
    \textbf{Discovery} & Same as described for BANK-APP. \\
    \textbf{Likelihood} & Same as described for BANK-APP. \\
    \textbf{Impact} & Same as described for BANK-APP. \\
    \textbf{Recommen\-dations} & Same as described for BANK-APP. \\ \hline
    \textbf{CVSS} & Same as described for BANK-APP.
    \\ \hline
\end{longtable}

\subsubsection{Comparison}
Though SecureBank restricts the entry of negative amounts while performing transactions, there is no contraint on 0.00 values. Both applications are vulnerable in this aspect.
\clearpage