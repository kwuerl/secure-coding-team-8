\subsection{Testing for Account Enumeration and Guessable User Account - OTG-IDENT-004} \label{OTG-IDENT-004}

\paragraph{BANK-APP} \mbox{}
\begin{longtable*}{p{.20\textwidth} | p{.80\textwidth}}
    \hline
    & \textbf{BANK-APP} \\
    \hline
    \textbf{Observation} &
	 It was found that the application responds with the same error messages for every client request that produces a failed authentication. This has been tested in the Login page.
    \\\\
    \textbf{Discovery} &
       This test was performed manually by trying various combinations of email and password. Steps are as follows:
       \begin{itemize}
       \item \underline{\textbf{Scenario 1 -}} - Testing for Valid user with right password
       		\begin{itemize}
       		 \item Open the Login page and enter a valid email and password.
       		 
       		 \item Click on the Submit button.
       		 
       		 \item The user is redirected to the Transactions page without any success message.
       		\end{itemize}
        \item \underline{\textbf{Scenario 2 -}} Testing for Valid user with wrong password
        	\begin{itemize}
        	  \item Open the Login page and enter a valid email with an incorrect password.
        	  
        	  \item Click on the Submit button.
        	  
        	  \item An error message is displayed that reads \textit{"Invalid Login Credentials"}, and the user stays on Login page. Also, the data entered in the form is cleared off. 
        	\end{itemize}
        	
       \item  \underline{\textbf{Scenario 3-}} Testing for non-existent User
	       \begin{itemize}
	       \item Open the Login page and enter an incorrect email and password.
	       
	       \item  Click on the Submit button.
	       
	       \item An error message is displayed that reads \textit{"Invalid Login Credentials"}, and the user stays on Login page. Also, the data entered in the form is cleared off.
	       \end{itemize}
       \end{itemize}
     \\\\
    \textbf{Likelihood} &
        Likelihood is high since this does not require any additional knowledge and can be done via Brute Force technique.
        It is also exploitable remotely.
    \\\\
    \textbf{Impact} &
        If this vulnerability existed, the responses to various incorrect requests would be different - similiar to "Incorrect password for user-xxx" or "User xxx does not exist" etc. This would have made it possible for the attacker to draw conclusions that a particular user does not exist, or that a user exists but with a different password. This could further lead to the attacker to enumerate existing users by guessing the login credentials.
    \\\\
    \textbf{CVSS} &
      \begin{tabular}{| l | l |}
      \hline
      Attack Vector		& \textcolor{red}{Network}\\
      \hline
      Attack Complexity	& \textcolor{red}{Low} \\
      \hline
      Privileges Required & \textcolor{red}{None} \\
      \hline
      User Interaction	& \textcolor{red}{None} \\
      \hline
      Scope		& \textcolor{Green}{Unchanged} \\
      \hline
      Confidentiality Impact	& \textcolor{red}{High} \\
      \hline
      Integrity Impact		& \textcolor{red}{High} \\
      \hline
      Availability Impact		& \textcolor{red}{High} \\
      \hline
      \end{tabular}
    \\
    \hline
\end{longtable*}
\paragraph{SecureBank} \mbox{}
\begin{longtable*}{p{.20\textwidth} | p{.80\textwidth}}
    \hline
    & \textbf{SecureBank} \\
    \hline
    \textbf{Observation} &
       Same as described for BANK-APP.
    \\\\
    \textbf{Discovery} &
    The behavior is similar in our application as well; and the vulnerability cannot be exploited as the error messages are consistent for all incorrect requests.
    \\\\
    \textbf{Likelihood} &
        N/A
    \\\\
    \textbf{Impact} &
        N/A
    \\\\
    \textbf{CVSS} &
        N/A
    \\
    \hline
\end{longtable*}
\clearpage