\subsection{Testing for Account Enumeration and Guessable User Account - OTG-IDENT-004}

\subsubsection{Observation}
It was found that the application responds with the same error messages for every client request that produces a failed authentication. This has been tested in the Login page.

\subsubsection{Discovery}
This test was performed manually by trying various combinations of email and password. \\
\\Steps: \\
\underline{\textbf{Scenario 1:}} Testing for Valid user with right password \\
\\1. Open the Login page and enter a valid email and password. \\
2. Click on the Submit button. \\
3. The user is redirected to the Transactions page without any success message. \\
\\\underline{\textbf{Scenario 2:}} Testing for Valid user with wrong password \\
\\1. Open the Login page and enter a valid email with an incorrect password. \\
2. Click on the Submit button. \\
3. An error message is displayed that reads \textit{"Invalid Login Credentials"}, and the user stays on Login page. Also, the data entered in the form is cleared off.\\
\\\underline{\textbf{Scenario 3:}} Testing for non-existent User\\
\\1. Open the Login page and enter an incorrect email and password. \\
2. Click on the Submit button. \\
3. An error message is displayed that reads \textit{"Invalid Login Credentials"}, and the user stays on Login page. Also, the data entered in the form is cleared off.\\

\subsubsection{Likelihood}
Likelihood is high since this does not require any additional knowledge and can be done via Brute Force technique.
It is also exploitable remotely.

\subsubsection{Impact}
If this vulnerability existed, the responses to various incorrect requests would be different - similiar to "Incorrect password for user-xxx" or "User xxx does not exist" etc. This would have made it possible for the attacker to draw conclusions that a particular user does not exist, or that a user exists but with a different password. This could further lead to the attacker to enumerate existing users by guessing the login credentials.

\subsubsection{CVSS}
\begin{tabular}{l | l}
Attack Vector		& Network \\
Attack Complexity	& Low \\
Privileges Required & None \\
User Interaction	& None \\
Scope				& None \\
Confidentiality		& High \\
Integrity			& High \\
Availability		& High
\end{tabular}

\subsubsection{Comparison with our application}
The behavior is similar in our application as well; and the vulnerability cannot be exploited as the error messages are consistent for all incorrect requests.

\subsubsection{Recommendation}
NA

\clearpage