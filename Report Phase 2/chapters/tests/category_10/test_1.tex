\subsection{Testing for DOM based Cross Site Scripting - OTG-CLIENT-001}
\subsubsection{BANK-APP}
\begin{longtable}[l]{ p{2.3cm} | p{.79\linewidth} }\hline
    & \textbf{BANK-APP}
    \\ \hline
    \textbf{Observation} & DOM based XSS uses the DOM present in the source as injection points. We tried to manipulate URLs to explore this vulnerability. However, no criticality was detected. \\
    \textbf{Discovery} &
         No tools were needed to discover this vulnerability. The URLs were modified and appended with script tags. But the response from the server did not reflect changes based on script tag. Steps are as follows:
         \begin{itemize}
             \item  Go to the Transactions page by entering the URL \code{http://<IP-address>/secure-coding/public/view \allowbreak \_transactions.php}.

             \item Append \code{\#<script>alert('hi')</script>} after the URL. After refreshing this page with this value, no change can be observed. Hence we can conclude that DOM based XSS is not found.
         \end{itemize}
    \\
    \textbf{Likelihood} & N/A \\
    \textbf{Impact} & N/A \\
    \textbf{Recommen\-dations} & N/A \\ \hline
    \textbf{CVSS} & N/A
    \\ \hline
\end{longtable}
\clearpage

\subsubsection{SecureBank}
\begin{longtable}[l]{ p{2.3cm} | p{.79\linewidth} }\hline
    & \textbf{SecureBank}
    \\ \hline
    \textbf{Observation} & DOM based XSS uses the DOM present in the source as injection points. We tried to manipulate URLs to explore this vulnerability. However, no criticality was detected. \\
    \textbf{Discovery} & Same as described for BANK-APP.\\
    \textbf{Likelihood} & N/A \\
    \textbf{Impact} & N/A \\
    \textbf{Recommen\-dations} & N/A \\ \hline
    \textbf{CVSS} & N/A
    \\ \hline
\end{longtable}

\subsubsection{Comparison}
Neither applications contain this vulnerability and behave similarly to the tests performed.
\clearpage