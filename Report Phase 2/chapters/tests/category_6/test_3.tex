\subsection{Testing for HTTP Verb Tampering - OTG-INPVAL-003}

\subsubsection{Observation}
It was observed that Verb Tampering could be done with HTTP requests but no critical vulnerability was exposed with it.\\
\\Methods that were allowed :\\
1. GET \\
2. POST \\
3. PUT \\
4. HEAD \\
5. OPTIONS \\
6. DELETE \\
\\Methods that were rejected \\
1. TRACE \\
2. CONNECT \\
\\With HEAD requests, there were no response data shown. In case of TRACE and CONNECT , the requests were rejected because of Same Origin Security restriction.


\subsubsection{Discovery}
Advanced Rest Client, an extension for the Google Chrome browser, was used to perform HTTP Verb Tampering.\\
\\Steps: \\
1. Open the extension in Chrome.\\
2. Enter the URL to be tested.\\
3. Then one select the type of request (GET, POST, PUT, PATCH, DELETE, HEAD, OPTIONS). Based on the type of HTTP request other details can be filled.\\
4. Click on the Send button.\\
\\Response can be seen in the lower section which helps in determining the criticality of the tampering done.

\subsubsection{Likelihood}
The exploitation of this vulnerability requires knowledge of basic tools to modify the server requests and the likelihood is low.

\subsubsection{Impact}
NA

\subsubsection{CVSS}
\begin{tabular}{l | l}
Attack Vector		& Network \\
Attack Complexity	& Low \\
Privileges Required & None \\
User Interaction	& None \\
Scope				& Unchanged \\
Confidentiality		& None \\
Integrity			& None \\
Availability		& None
\end{tabular}

\subsubsection{Comparison with our application}
Verb Tampering in our application is also possible but without any vulnerability being exposed to the attacker.

\subsubsection{Recommendation}
The server settings should be modified to cater only to request types that are used in the application; in this case - GET and POST.
\clearpage