\subsection{Testing for Account Enumeration and Guessable User Account - OTG-IDENT-004} \label{OTG-IDENT-004}
\subsubsection{InternetBanking}
\begin{longtable}[l]{ p{2.3cm} | p{.79\linewidth} }\hline
    & \textbf{InternetBanking} \\ \hline
    \textbf{Observation} & It was found that the application responds with the same error messages for every client request that produces a failed authentication. This has been tested in the Login page. \\
    \textbf{Discovery} &
         This test was performed manually by trying various combinations of email and password. Steps are as follows:
            \begin{itemize}
            \item \textbf{Scenario 1 -} Testing for Valid user with right password
            		\begin{itemize}
            		 \item Open the Login page and enter a valid email and password. Click on the Submit button.

            		 \item The user is redirected to the Transactions page without any success message.
            		\end{itemize}
             \item \textbf{Scenario 2 -} Testing for Valid user with wrong password
             	\begin{itemize}
             	  \item Open the Login page and enter a valid email with an incorrect password. Click on the Submit button.

             	  \item An error message is displayed that reads \code{
             	  Login and/or password not correct.}, and the user stays on Login page. Also, the data entered in the form is cleared off.
             	\end{itemize}

            \item \textbf{Scenario 3-} Testing for non-existent User
     	       \begin{itemize}
     	       \item Open the Login page and enter an incorrect email and password. Click on the Submit button.

     	       \item An error message is displayed that reads \code{
     	       Login and/or password not correct.}, and the user stays on Login page. Also, the data entered in the form is cleared off.
     	       \end{itemize}
            \end{itemize}
    \\
    \textbf{Likelihood} & N/A \\
    \textbf{Impact} & N/A \\
    \textbf{Recommen\-dations} & N/A \\ \hline
    \textbf{CVSS} & N/A
    \\ \hline
\end{longtable}

\subsubsection{SecureBank}
\begin{longtable}[l]{ p{2.3cm} | p{.79\linewidth} }\hline
    & \textbf{SecureBank} \\ \hline
    \textbf{Observation} & It was found that the application responds with different error messages for different incorrect login requests. However, it is not possible to enumerate users. There is also a lockout mechanism which locks the account after 5 failed login attempts.\\
    \textbf{Discovery} &
        This test was performed manually by trying various combinations of email and password. Steps are as follows:
            \begin{itemize}
            \item \textbf{Scenario 1 -} Testing for Valid user with right password
                    \begin{itemize}
                     \item Open the Login page and enter a valid email and password. Click on the Submit button.

                     \item The user is redirected to the Home page without any success message.
                    \end{itemize}
             \item \textbf{Scenario 2 -} Testing for Valid user with wrong password
                \begin{itemize}
                  \item Open the Login page and enter a valid email with an incorrect password. Click on the Submit button.

                  \item An error message is displayed that reads \code{
                  Login failed - Either the e-mail or the password is wrong.}, and the user stays on Login page. Also, the data entered in the form is retained.
                \end{itemize}

            \item \textbf{Scenario 3-} Testing for non-existent User
               \begin{itemize}
               \item Open the Login page and enter an incorrect email and password. Click on the Submit button.

               \item An error message is displayed that reads \code{
               Login failed - There is no account with this email.}, and the user stays on Login page. Also, the data entered in the form is retained.
               \end{itemize}
            \end{itemize}
    \\
    \textbf{Likelihood} & N/A \\
    \textbf{Impact} & N/A \\
    \textbf{Recommen\-dations} & N/A \\ \hline
    \textbf{CVSS} & N/A
    \\ \hline
\end{longtable}

\subsubsection{Comparison}
In InternetBanking application, the vulnerability cannot be exploited as the error messages are consistent for all incorrect requests.
In case of SecureBank, though the error messages differ, no vulnerability has been detected owing to the absence of username-baseed login and the presence of a lockout mechanism.
Hence, neither of the applications contain any vulnerabilty.
\clearpage