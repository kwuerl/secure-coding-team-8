\subsection{Testing for Browser cache weakness - OTG-AUTHN-006}

\begin{longtable}[l]{ p{2.3cm} | p{.79\linewidth} }\hline
    & \textbf{InternetBanking} \\ \hline
    \textbf{Observation} & No settings related to caching were found in the server configuration. However, the HTTP responses contained the header \code{Cache-Control: must-revalidate, pre-check=0, post-check=0, no-store, no-cache}, which relates to the Apache module \code{mod\_expires}. It is also found that sensitive data is not saved anywhere throughout the application. \\
    \textbf{Discovery} & Manual inspection of \code{.htaccess} file and the Apache configuration was done. \\
    \textbf{Likelihood} & N/A \\
    \textbf{Impact} & N/A \\
    \textbf{Recommen\-dations} & N/A \\ \hline
    \textbf{CVSS} & N/A
    \\ \hline
\end{longtable}

\begin{longtable}[l]{ p{2.3cm} | p{.79\linewidth} }\hline
    & \textbf{SecureBank} \\ \hline
    \textbf{Observation} & No settings related to caching were found in the server configuration. However, the HTTP responses contained the header \code{Cache-Control: must-revalidate, pre-check=0, post-check=0, no-store, no-cache}, which relates to the Apache module \code{mod\_expires}. It is also found that sensitive data is not saved anywhere throughout the application. \\
    \textbf{Discovery} & Manual inspection of \code{.htaccess} file and the Apache configuration was done. \\
    \textbf{Likelihood} & Same as described for InternetBanking. \\
    \textbf{Impact} & N/A \\
    \textbf{Recommen\-dations} & N/A \\ \hline
    \textbf{CVSS} & N/A
    \\ \hline
\end{longtable}

\subsubsection{Comparison}
Both of the applications behave similarly in this case and are secure since no sensitive data is stored in browser cache.
\clearpage