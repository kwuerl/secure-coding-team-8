\subsection{Analysis of Error Codes - OTG-ERR-001}
\subsubsection{Online Banking}
\begin{longtable}{ p{2.3cm} | p{.79\linewidth} }\hline
    & \textbf{Online Banking}
    \hfill CVSS Score: 4.3 \progressbar[filledcolor=BurntOrange]{0.43}
    \\ \hline
    \textbf{Observation} & 
    	Error messages:
    	\begin{itemize}
		  \item \textbf{Application Errors:} 
		  	\begin{itemize}
			  \item The Application uses Custom Error messages for most of its error feedback
			  \item MySQL Errors are tried to be presented as \code{mysql_errno()}. Unfortunately this is a invalid syntax.
			  \item The Application presents direct shell output to the user.
              \item There are many cases where Exeptions are not catched correctly
			\end{itemize}
		\end{itemize}
    \\
    \textbf{Discovery} &
    	On most errors the Application responds with \code{die(message);}.
        In most cases in \code{DataAccess.php} the Syntax used (\code{die(mysql_errno())}) is wrong and causes a php syntax error.
        Other errors are thrown as Exception with custom message.
        Many of these Exceptions are not caught later.
        For Example the Exception Thrown in \code{DataAccess.php} line 536 is never caught.
        Due to the current configuration of the Webserver these Exceptions do not end up on the users screen but are only posted to the log file. But a Webserver that is not configured to suppress these Exceptions and Error output would allow to expose a great ammount of Stack traces and Exception codes.
    \\
    \textbf{Likelihood} &
    	An Atacker can use the informations presented by the error messages to validate further atacks. This results in a higher likelihood for other atacks.
    \\
    \textbf{Impact} & 
    	The Mysql error codes can be used by an atacker to directly verify the success of his actions and help him to correct errors.
    \\
    \textbf{Recommen\-dations} &
        Catch Exceptions and hide the direct errors and transalte them to more general custom error messages.
        Try to avoid incorrect php syntax in code parts that handle errors.
    \\ \hline
    \textbf{CVSS} &
        \begin{tabular}[t]{@{}l | l}
            Attack Vector           & \textcolor{red}{Network} \\
            Attack Complexity       & \textcolor{red}{Low} \\
            Privileges Required     & \textcolor{BurntOrange}{Low} \\
            User Interaction        & \textcolor{red}{None} \\
            Scope                   & \textcolor{Green}{Unchanged} \\
            Confidentiality Impact  & \textcolor{BurntOrange}{Low} \\
            Integrity Impact        & \textcolor{Green}{None} \\
            Availability Impact     & \textcolor{Green}{None}
        \end{tabular}
    \\ \hline
\end{longtable}
\clearpage

\subsubsection{SecureBank}
\begin{longtable}{ p{2.3cm} | p{.79\linewidth} }\hline
    & \textbf{SecureBank}
    \\ \hline
    \textbf{Observation} & 
    	No direct PHP Error messages are passed on to the user. See \ref{code:error_routing_sb}
    \\
    \textbf{Discovery} &
    	N/A
    \\
    \textbf{Likelihood} & 
    	N/A
    \\
    \textbf{Impact} & 
    	N/A
    \\
    \textbf{Recommen\-dations} &
        N/A
    \\ \hline
    \textbf{CVSS} &
        N/A
    \\ \hline
\end{longtable}

\subsubsection{Comparison}
Online Banking has some severe flaws in Syntax and Programm Logic.

\clearpage

\begin{lstlisting}[caption={PHP code for error handling from RoutingService.php}\label{code:error_routing_sb}, language=PHP, basicstyle=\footnotesize, frame=single, captionpos=t, linewidth=.9\textwidth, xleftmargin=.12\textwidth]
/**
 * Runs the callback for the given request
 */
public function dispatch()
{
    try {

            ...

    } catch (\Exception $e) {
        if(_DEBUG === true) throw $e;
        if (!array_key_exists("503", $this->error_callbacks)) {
            $this->error_callbacks["503"] = function() {
                header($_SERVER['SERVER_PROTOCOL']." 503 Fail...");
                echo '503';
            };
        }
        call_user_func($this->error_callbacks["503"]);
        return;
    }
}
\end{lstlisting}
