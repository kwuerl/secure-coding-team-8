\subsection{Testing for Stored Cross Site Scripting - OTG-INPVAL-002} \label{OTG-INPVAL-002}
\begin{longtable}[l]{ p{2.3cm} | p{.79\linewidth} }\hline
    & \textbf{Online Banking}
    \\ \hline
    \textbf{Observation} & It was found that it is not possible perform stored XSS in the application. Simple HTML and Script tags were tried and but the attacks were unsuccessful. \\
    \textbf{Discovery} & Manual inspection of code revealed usage of functions such as \code{htmlspecialchars}, \code{filter\_var} and \code{preg\_match} for input sanitization; leading to HTML tags being rendered as plain text. \\
    \textbf{Likelihood} & N/A\\
    \textbf{Impact} & N/A \\
    \textbf{Recommen\-dations} & N/A \\ \hline
    \textbf{CVSS} & N/A
    \\ \hline
\end{longtable}

\begin{longtable}[l]{ p{2.3cm} | p{.79\linewidth} }\hline
    & \textbf{SecureBank}
    \\ \hline
    \textbf{Observation} & It was found that it is not possible to perform stored XSS in the application. Simple HTML and Script tags were tried and the attacks were unsuccessful. \\
    \textbf{Discovery} & Manual inspection of code revealed usage of functions such as \code{htmlspecialchars}, \code{filter\_var} and \code{preg\_match} for input sanitization; leading to HTML tags being rendered as plain text. \\
    \textbf{Likelihood} & N/A \\
    \textbf{Impact} & N/A \\
    \textbf{Recommen\-dations} & N/A \\ \hline
    \textbf{CVSS} & N/A
    \\ \hline
\end{longtable}

\subsubsection{Comparison}
Both applications are secure with respect to Stored XSS attacks.
\clearpage