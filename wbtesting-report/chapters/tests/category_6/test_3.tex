\subsection{Testing for HTTP Verb Tampering - OTG-INPVAL-003}

\begin{longtable}[l]{ p{2.3cm} | p{.79\linewidth} }\hline
    & \textbf{Online Banking}
    \\ \hline
    \textbf{Observation} & It was observed that Verb Tampering could be done with HTTP requests but no critical vulnerability was exposed with it. Methods that were allowed :
           \begin{itemize}
     	      \item  GET
     	      \item  POST
     	      \item  HEAD
     	      \item OPTIONS
           \end{itemize}
           Methods that were rejected:
           \begin{itemize}
       	      \item  TRACE
       	      \item  CONNECT
           \end{itemize}
           With HEAD requests, there were no response data shown. In case of \code{TRACE} and \code{CONNECT}, the requests were rejected because of Same Origin Security restriction.
    \\
    \textbf{Discovery} & It was found that there is no reference to methods other than GET and POST, in the source code by scanning the code using \code{grep}.
    \\
    \textbf{Likelihood} & N/A \\
    \textbf{Impact} & N/A \\
    \textbf{Recommen\-dations} & N/A \\ \hline
    \textbf{CVSS} & N/A
    \\ \hline
\end{longtable}

\begin{longtable}[l]{ p{2.3cm} | p{.79\linewidth} }\hline
    & \textbf{SecureBank}
    \\ \hline
    \textbf{Observation} & It has been observed that Verb Tampering is possible but without any vulnerability being exposed to the attacker.\\
    \textbf{Discovery} & Same as observed for Online Banking. \\
    \textbf{Likelihood} & N/A \\
    \textbf{Impact} & N/A \\
    \textbf{Recommen\-dations} & N/A\\ \hline
    \textbf{CVSS} & N/A
    \\ \hline
\end{longtable}

\subsubsection{Comparison}
Neither application exposes vulnerability though verb tampering is possible. Hence both seem secure.
\clearpage