\chapter{Overview of most important observations}

\section{Vulnerabilities of Online Banking}

\subsection{Sensitive Data Exposure}
There is no implementation of HTTPS. So sensitive data is communicated without any encryption.
\begin{itemize}
	\item \textbf{Likelihood:} High
	\item \textbf{Impact:} High
	\item \textbf{Risk:} High
	\item \textbf{Reference:} OTG-AUTHN-001
\end{itemize}

\subsection{Session Hijacking}
The session cookie is not set to Secure or HttpOnly, thus allowing manipulation from client-side.
\begin{itemize}
	\item \textbf{Likelihood:} High
	\item \textbf{Impact:} High
	\item \textbf{Risk:} High
	\item \textbf{Reference:} OTG-SESS-001 and OTG-SESS-003
\end{itemize}

\subsection{Weak Password Policy}
Passwords are stored in database using md5, thus easily traceable. No password policy during password reset.
\begin{itemize}
	\item \textbf{Likelihood:} High
	\item \textbf{Impact:} High
	\item \textbf{Risk:} High
	\item \textbf{Reference:} OTG-AUTHN-007
\end{itemize}

\subsection{Weak lockout mechanisms}
There are no weak lockout mechanisms available. Therefore account bruteforcing is possible.
\begin{itemize}
	\item \textbf{Likelihood:} High
	\item \textbf{Impact:} High
	\item \textbf{Risk:} High
	\item \textbf{Reference:} OTG-AUTHN-003
\end{itemize}
%%%%%%%%

\section{Vulnerabilities of SecureBank}

\subsection{Test Number of Times a Function Can be Used Limits}
after 100 transactions no new tan codes are sent to the user and this way he can not use the application any more.\\
\begin{itemize}
	\item \textbf{Likelihood:} High
	\item \textbf{Impact:} Low
	\item \textbf{Risk:} Low
	\item \textbf{Reference:} OTG-BUSLOGIC-006
\end{itemize}